\section{Haskell: História e Evolução}

Haskell é uma \textbf{linguagem de programação puramente funcional}, de \textbf{tipagem forte} e \textbf{avaliação preguiçosa}. Criada no final dos anos 1980, seu nome homenageia o matemático \textbf{Haskell Curry}, cujas pesquisas influenciaram a lógica combinatória e a programação funcional.

Desenvolvida com o objetivo de ser uma \textbf{linguagem padrão para pesquisa em programação funcional}, Haskell se tornou um ambiente para experimentação e inovação em teoria de linguagens de programação.

\subsection{Origens e Motivação}

Na década de 1980, houve um crescente interesse em \textbf{linguagens funcionais}, mas também uma fragmentação entre diferentes dialetos, como \textbf{Miranda, ML e Lisp}. Para unificar esse cenário, em \textbf{setembro de 1987}, durante a \textbf{Conferência sobre Linguagens Funcionais e Arquiteturas Computacionais (FPCA '87)}, um grupo de pesquisadores decidiu criar uma \textbf{linguagem funcional padronizada}. Assim, nasceu o projeto \textbf{Haskell}.

O principal objetivo era combinar os \textbf{melhores aspectos das linguagens funcionais existentes}, facilitando tanto a \textbf{pesquisa acadêmica} quanto o \textbf{desenvolvimento de aplicações reais}.

\subsection{Desenvolvimento e Principais Marcos}

\textbf{Linha do tempo do Haskell:}

\begin{itemize}
    \item \textbf{1987} → Proposta inicial da linguagem na FPCA '87.
    \item \textbf{1990} → Lançamento do \textbf{Haskell 1.0}, estabelecendo a base da linguagem com \textbf{avaliação preguiçosa} e um \textbf{sistema de tipos avançado}.
    \item \textbf{1991-1996} → Atualizações frequentes até o \textbf{Haskell 1.4}, refinando a \textbf{sintaxe, sistema de módulos e inferência de tipos}.
    \item \textbf{1997} → Publicação do \textbf{Haskell 98}, uma versão estável para \textbf{ensino e desenvolvimento}.
    \item \textbf{2001} → Lançamento do \textbf{GHC (Glasgow Haskell Compiler)}, o principal compilador de Haskell, ampliando sua adoção.
    \item \textbf{2003} → Criação do \textbf{Haskell 2010}, uma versão revisada com melhorias e remoção de recursos obsoletos.
    \item \textbf{2014} → Haskell ganha destaque na indústria, sendo adotado em áreas como \textbf{finanças, segurança e sistemas distribuídos}.
    \item \textbf{2020} → Discussões sobre o \textbf{Haskell 2020}, focando em tornar a linguagem mais \textbf{acessível e moderna}.
\end{itemize}

\subsection{Crescimento e Influência}

\textbf{Haskell influenciou diversas linguagens modernas}, incluindo:

\begin{itemize}
    \item \textbf{Scala}
    \item \textbf{Rust}
    \item \textbf{F#}
    \item \textbf{PureScript}
\end{itemize}

Seus conceitos de \textbf{avaliação preguiçosa e sistema de tipos robusto} ajudaram a moldar linguagens que combinam \textbf{paradigmas funcionais e imperativos}.

Além do uso acadêmico, \textbf{grandes empresas} adotaram Haskell para ferramentas internas, análise de dados e segurança, como:

\begin{itemize}
    \item \textbf{Facebook}
    \item \textbf{Google}
    \item \textbf{Standard Chartered}
\end{itemize}

Projetos populares como o \textbf{GHC (compilador de Haskell)} e frameworks como \textbf{Yesod} também contribuíram para sua popularidade.

\subsection{Dicionário de Termos}

Aqui estão alguns conceitos fundamentais para entender Haskell:

\begin{itemize}
    \item \textbf{Funcional} → Paradigma de programação onde funções são \textbf{cidadãos de primeira classe}, ou seja, podem ser passadas como argumentos e retornadas como valores. A ênfase está na \textbf{imutabilidade e composição de funções}.
    \item \textbf{Tipagem Forte} → Sistema de tipos que \textbf{não permite conversões implícitas inseguras} entre diferentes tipos de dados. Isso reduz erros e torna o código mais seguro.
    \item \textbf{Avaliação Preguiçosa} → Estratégia onde as expressões só são avaliadas \textbf{quando realmente necessárias}, o que pode melhorar o desempenho e permitir a criação de \textbf{estruturas de dados infinitas}.
    \item \textbf{Inferência de Tipos} → Capacidade do compilador de \textbf{deduzir automaticamente os tipos} das expressões sem necessidade de anotação explícita.
    \item \textbf{Imutabilidade} → Dados não podem ser modificados após a criação, evitando efeitos colaterais e tornando o código mais previsível e confiável.
\end{itemize}

\subsection{Impacto e Perspectivas Futuras}

Haskell, que começou como um projeto acadêmico, se consolidou como uma das linguagens funcionais mais influentes na história da computação. Seu impacto transcende as fronteiras da pesquisa acadêmica, refletindo-se em uma adaptação crescente na indústria tecnológica. Através de sua abordagem inovadora com tipagem forte, avaliação preguiçosa e sistemas de tipos robustos, Haskell continua a moldar o desenvolvimento de software, especialmente em áreas que exigem alta confiabilidade e desempenho. Com uma comunidade ativa e dedicada, a linguagem segue em constante evolução, explorando novas possibilidades para a programação funcional e declarativa, e permanece relevante tanto para acadêmicos quanto para profissionais da indústria.
