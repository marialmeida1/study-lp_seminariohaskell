\section{Introdução}

A computação está em constante evolução, mas seus princípios fundamentais se mantêm consistentes, mesmo diante de novas inovações \cite{Thompson2011}. No caso do hardware, as arquiteturas desenvolvidas no século passado ainda se baseiam em conceitos que permanecem amplamente inalterados. Em contraste, o software experimentou uma evolução mais gradual, com novos paradigmas demorando a conquistar aceitação no mercado.

O paradigma de programação funcional, embora não seja recente, demorou a alcançar ampla adoção. Já na década de 1950, John McCarthy introduziu a linguagem Lisp, a primeira a ser classificada como funcional e a segunda mais antiga na história da computação \cite{Hutton2007}. Contudo, foi apenas com o advento da linguagem Haskell que a programação funcional se estabeleceu como uma ferramenta relevante em áreas como finanças, criptomoedas, blockchain e segurança — campos nos quais a confiabilidade, a segurança e a matemática desempenham papéis essenciais.

Haskell é uma linguagem funcional pura, fortemente tipada e com um alto grau de abstração, o que a torna uma excelente ferramenta para expressar conceitos matemáticos de maneira direta e elegante. A programação funcional, ao contrário do modelo imperativo, evita a mutabilidade das variáveis. Em Haskell, as variáveis são imutáveis, o que significa que uma vez atribuídos valores a elas, estes não podem ser alterados. Isso reflete a abordagem matemática, onde as variáveis não mudam seu valor durante o processo de cálculo, mas representam constantes definidas.

Além disso, Haskell não utiliza a estrutura de "loops" típicos de outras linguagens. Ao invés de iterar sobre valores e alterar seu estado a cada ciclo, Haskell permite que o resultado seja obtido por meio da composição de funções. Este comportamento é uma manifestação direta da matemática, onde expressões não são alteradas enquanto são resolvidas, mas sim avaliadas com base em suas definições e propriedades.

Essa distinção na abordagem de fluxo de controle é um dos maiores contrastes entre a programação funcional e o paradigma iterativo. Enquanto em linguagens imperativas o fluxo do programa é determinado pela modificação de variáveis e pela repetição de operações, em Haskell a construção do programa se dá de maneira não sequencial. O programador define as funções que, uma vez compostas, produzem o resultado final de forma mais declarativa, refletindo a definição matemática do problema.

A relevância de Haskell vai além de sua utilidade prática no mercado de tecnologia. Seu impacto intelectual é profundo, influenciando o desenvolvimento de novas abordagens e paradigmas de programação. A linguagem serve de modelo para tecnologias emergentes, onde a matemática e a segurança são essenciais. Ao alinhar conceitos matemáticos com implementações de software, Haskell contribui para uma melhor integração entre hardware e software, possibilitando a criação de soluções mais robustas e confiáveis. Neste artigo, exploraremos mais sobre a linguagem Haskell, suas principais características e o impacto que ela tem na evolução da computação.
