\section{Linguagens Correlatadas}

Linguagens correlatadas são aquelas que compartilham \textbf{conceitos, paradigmas ou influências} entre si. Elas podem ter evoluído a partir de uma base teórica comum ou sido inspiradas por linguagens anteriores para resolver problemas semelhantes.

No caso do \textbf{Haskell}, várias linguagens funcionais e multiparadigma foram influenciadas por ele ou compartilharam ideias semelhantes ao longo do tempo. Isso demonstra a importância de Haskell no avanço da computação funcional e no desenvolvimento de sistemas de tipos sofisticados.

\subsection{O que são linguagens correlatadas?}

Linguagens correlatadas podem ser classificadas de diferentes formas:

\begin{itemize}
    \item \textbf{Linguagens predecessoras} $\rightarrow$ Linguagens que inspiraram o desenvolvimento de uma nova linguagem, fornecendo ideias fundamentais.
    \item \textbf{Linguagens sucessoras} $\rightarrow$ Linguagens que evoluíram ou foram fortemente influenciadas por outra, incorporando ou aprimorando seus conceitos.
    \item \textbf{Linguagens com paradigmas semelhantes} $\rightarrow$ Linguagens que compartilham características importantes, como imutabilidade, tipagem forte ou avaliação preguiçosa, mesmo que não sejam diretamente influenciadas umas pelas outras.
\end{itemize}

Haskell, por exemplo, tem fortes conexões com outras linguagens funcionais e influenciou novas linguagens ao longo dos anos.

\subsection{Linguagens correlatadas ao Haskell}

Aqui estão algumas das linguagens mais relacionadas ao Haskell, seja por influência direta ou por compartilharem características comuns:

\subsubsection{Linguagens que influenciaram o Haskell}
\begin{itemize}
    \item \textbf{Miranda} $\rightarrow$ Linguagem funcional baseada em avaliação preguiçosa, foi uma grande influência para a criação do Haskell, principalmente por sua simplicidade e eficiência na manipulação de funções.
    \item \textbf{ML (MetaLanguage)} $\rightarrow$ Influenciou o sistema de tipos do Haskell, especialmente a inferência de tipos e o uso de polimorfismo paramétrico.
    \item \textbf{Lisp} $\rightarrow$ Embora seja uma linguagem mais antiga e com um paradigma distinto, ajudou a estabelecer conceitos como \textbf{funções de primeira classe}, \textbf{recursão} e \textbf{tratamento simbólico}.
    \item \textbf{Scheme} $\rightarrow$ Subconjunto do Lisp que introduziu simplificações e conceitos que também aparecem no Haskell, como imutabilidade e funções puras.
\end{itemize}

\subsubsection{Linguagens influenciadas pelo Haskell}
\begin{itemize}
    \item \textbf{F#} $\rightarrow$ Inspirada no Haskell e no ML, mas integrada ao ecossistema .NET, permitindo programação funcional dentro do ambiente Microsoft.
    \item \textbf{Scala} $\rightarrow$ Embora seja multiparadigma, incorpora muitas características funcionais influenciadas pelo Haskell, como inferência de tipos e imutabilidade por padrão.
    \item \textbf{Rust} $\rightarrow$ Adota um sistema de tipos forte e segurança de memória semelhante ao que Haskell já implementava, além de permitir programação funcional dentro de um contexto de sistemas.
    \item \textbf{PureScript} $\rightarrow$ Uma linguagem funcional fortemente tipada para desenvolvimento web, semelhante ao Haskell, mas voltada para JavaScript e execução no navegador.
    \item \textbf{Elm} $\rightarrow$ Linguagem funcional voltada para desenvolvimento front-end que compartilha conceitos como imutabilidade e sistema de tipos robusto com Haskell.
    \item \textbf{Idris} $\rightarrow$ Linguagem funcional inspirada no Haskell, mas com suporte a \textbf{tipos dependentes}, o que permite maior expressividade e verificação de programas em tempo de compilação.
    \item \textbf{Agda} $\rightarrow$ Similar ao Idris, usada para \textbf{provas formais} e programação baseada em tipos dependentes, evoluindo a partir dos conceitos do Haskell.
\end{itemize}

\subsubsection{Linguagens com conceitos semelhantes}
\begin{itemize}
    \item \textbf{OCaml} $\rightarrow$ Variante do ML que compartilha a tipagem forte e inferência de tipos do Haskell, além de um sistema modular avançado.
    \item \textbf{Erlang} $\rightarrow$ Embora seja mais voltada para concorrência, compartilha a imutabilidade e algumas abordagens funcionais, especialmente na forma como lida com \textbf{processos e mensagens}.
    \item \textbf{Elixir} $\rightarrow$ Construída sobre a máquina virtual do Erlang, mantendo fortes conceitos funcionais como \textbf{composição de funções} e \textbf{imutabilidade}.
    \item \textbf{Coq} $\rightarrow$ Linguagem usada para verificação formal de programas, compartilhando ideias de tipagem forte e imutabilidade com Haskell.
    \item \textbf{Clojure} $\rightarrow$ Dialeto moderno do Lisp que adota imutabilidade por padrão, promovendo um estilo funcional semelhante ao do Haskell.
\end{itemize}

\subsection{Importância das linguagens correlatadas}

As linguagens correlatadas ajudam no avanço da programação, permitindo que conceitos bem-sucedidos sejam reutilizados e aprimorados. O Haskell, por exemplo, trouxe inovações como \textbf{avaliação preguiçosa, monads para modelagem de efeitos colaterais e um poderoso sistema de tipos}, que influenciaram diversas outras linguagens modernas.

\subsubsection{Principais benefícios das linguagens correlatadas:}
\begin{itemize}
    \item \textbf{Transferência de conhecimento} $\rightarrow$ Programadores que aprendem Haskell podem facilmente entender e trabalhar com linguagens similares, como F# e Scala.
    \item \textbf{Aprimoramento contínuo} $\rightarrow$ Linguagens novas incorporam os melhores aspectos de suas predecessoras, tornando-se mais eficientes e produtivas.
    \item \textbf{Exploração acadêmica} $\rightarrow$ Haskell tem sido um grande campo de pesquisa para novos conceitos em programação funcional e segurança de tipos.
    \item \textbf{Interoperabilidade} $\rightarrow$ Algumas dessas linguagens compartilham bibliotecas e frameworks, facilitando a integração entre diferentes tecnologias.
\end{itemize}

Ao entender essas conexões, desenvolvedores podem transitar entre diferentes linguagens com maior facilidade e escolher a melhor ferramenta para cada problema.

\subsection{Evolução Contínua}

Haskell não surgiu isoladamente, mas como parte de uma longa evolução das linguagens funcionais. Ele se baseou em ideias anteriores, como as de \textbf{Miranda e ML}, e serviu de inspiração para linguagens mais recentes, como \textbf{F#, Scala e PureScript}. Esse intercâmbio contínuo de ideias é o que impulsiona o progresso na ciência da computação.

Seja na academia ou na indústria, compreender essas conexões nos ajuda a \textbf{aproveitar melhor as ferramentas disponíveis} e a contribuir para a evolução das linguagens de programação.
